\documentclass[12pt]{article}
\usepackage{geometry}
\usepackage{graphicx}
\usepackage{setspace}
\usepackage{amsmath}
\usepackage{amssymb}
\usepackage{titlesec}
\usepackage{hyperref}
\geometry{margin=1in}
\setstretch{1.5}

\titleformat{\section}{\large\bfseries}{\thesection.}{0.5em}{}

\title{Synthesis of Magnetite (Fe\textsubscript{3}O\textsubscript{4}) Nanoparticles by Co-Precipitation Method and Their Structural Characterization}
\author{Ammar Yasir \\
BS Chemistry, University of Central Punjab, Lahore, Pakistan \\
\texttt{ammaryasir922@gmail.com} \\
\and
Muhammad Ahmad Raza \\
Assistant Professor, Department of Chemistry, \\
University of Central Punjab, Lahore, Pakistan \\
\texttt{ahmad.raza@ucp.edu.pk}
}
\date{}

\begin{document}
\maketitle

\section*{Abstract}
Magnetite nanoparticles (Fe\textsubscript{3}O\textsubscript{4}) have garnered significant interest due to their magnetic properties, chemical stability, and wide range of potential applications. This paper presents the synthesis of Fe\textsubscript{3}O\textsubscript{4} nanoparticles using the co-precipitation method, a simple and cost-effective technique. The structural characterization was carried out using X-ray diffraction (XRD), revealing the crystalline nature of the nanoparticles and confirming the formation of the Fe\textsubscript{3}O\textsubscript{4} phase. The average crystallite size was calculated using the Debye–Scherrer equation. The results indicate successful synthesis of magnetite nanoparticles with desirable properties for various technological and biomedical applications.

\textbf{Keywords:} Magnetite nanoparticles, co-precipitation method, XRD, Debye–Scherrer equation, structural characterization.

\section{Introduction}
Nanotechnology has emerged as a revolutionary field with applications across diverse domains, including medicine, electronics, and environmental remediation. Among various nanomaterials, magnetic nanoparticles, particularly magnetite (Fe\textsubscript{3}O\textsubscript{4}), have attracted significant attention owing to their unique magnetic properties, biocompatibility, and ease of surface modification.

The co-precipitation method is one of the most widely used techniques for synthesizing Fe\textsubscript{3}O\textsubscript{4} nanoparticles due to its simplicity, low cost, and ability to produce nanoparticles with controlled size and morphology. In this study, Fe\textsubscript{3}O\textsubscript{4} nanoparticles were synthesized using this method, and their structural properties were characterized using X-ray diffraction (XRD).

\section{Experimental}

\subsection{Materials}
All chemicals used in this study were of analytical grade and used without further purification. Iron (III) chloride hexahydrate (FeCl\textsubscript{3}·6H\textsubscript{2}O), iron (II) sulfate heptahydrate (FeSO\textsubscript{4}·7H\textsubscript{2}O), and sodium hydroxide (NaOH) were obtained from Sigma-Aldrich. Distilled water was used throughout the experiments.

\subsection{Synthesis of Fe\textsubscript{3}O\textsubscript{4} Nanoparticles}
Fe\textsubscript{3}O\textsubscript{4} nanoparticles were synthesized by the co-precipitation method. Aqueous solutions of FeCl\textsubscript{3}·6H\textsubscript{2}O and FeSO\textsubscript{4}·7H\textsubscript{2}O in a molar ratio of 2:1 were prepared and mixed under constant stirring at 80°C. Sodium hydroxide solution was added dropwise until the pH of the solution reached 10, resulting in the formation of a black precipitate. The precipitate was washed several times with distilled water and ethanol to remove impurities and then dried at 60°C overnight.

\section{Results and Discussion}

\subsection{X-Ray Diffraction (XRD) Analysis}
The crystalline structure of the synthesized Fe\textsubscript{3}O\textsubscript{4} nanoparticles was examined using XRD. The diffraction peaks corresponded to the standard patterns of magnetite, indicating the successful formation of Fe\textsubscript{3}O\textsubscript{4}. The peaks at 2$\theta$ values of approximately 30.1°, 35.5°, 43.1°, 53.4°, 57.0°, and 62.6° correspond to the (220), (311), (400), (422), (511), and (440) planes, respectively.

The average crystallite size (D) was calculated using the Debye–Scherrer equation:
\[
D = \frac{K\lambda}{\beta \cos \theta}
\]
where:
\begin{itemize}
    \item $D$ is the crystallite size,
    \item $K$ is the shape factor (typically 0.9),
    \item $\lambda$ is the X-ray wavelength (1.5406 Å for Cu K$\alpha$),
    \item $\beta$ is the full width at half maximum (FWHM) in radians,
    \item $\theta$ is the Bragg angle.
\end{itemize}

The average crystallite size was found to be approximately 12 nm, indicating the nanoscale nature of the synthesized particles.

\section{Conclusion}
Fe\textsubscript{3}O\textsubscript{4} nanoparticles were successfully synthesized using the co-precipitation method. XRD analysis confirmed the crystalline structure and phase purity of the nanoparticles. The calculated average crystallite size was around 12 nm, highlighting their potential for applications in various fields, including biomedical imaging, drug delivery, and environmental remediation.

\end{document}
